\documentclass[coursework, och]{SCWorks1}

\usepackage[utf8]{inputenc}
\usepackage{cmap}
\usepackage[T2A]{fontenc}
\usepackage{graphicx}

\usepackage[sort,compress]{cite}
\usepackage{amsmath}
\usepackage{amssymb}
\usepackage{amsthm}
\usepackage{fancyvrb}
\usepackage{longtable}
\usepackage{array}
\usepackage[english,russian]{babel}

\usepackage[colorlinks=false]{hyperref}

\renewcommand{\labelenumii}{\arabic{enumi}.\arabic{enumii}.}

\newcommand{\eqdef}{\stackrel {\rm def}{=}}

\newtheorem{lem}{Лемма}
\newtheorem{definition}{Определение}

\begin{document}


\begin{center}
    МИНОБРНАУКИ РОССИИ

    Федеральное государственное бюджетное образовательное учреждение высшего образования

    <<САРАТОВСКИЙ НАЦИОНАЛЬНЫЙ ИССЛЕДОВАТЕЛЬСКИЙ ГОСУДАРСТВЕННЫЙ УНИВЕРСИТЕТ ИМЕНИ Н.Г. ЧЕРНЫШЕВСКОГО>>
\end{center}

\vspace{10pt}

\begin{center}
Кафедра системного анализа и автоматического управления
\end{center}

\vspace{50pt}

\begin{center}
    \textbf{Отчет по лабораторной работе 1.}
\end{center}

\vspace{20pt}

\begin{flushleft}
Студента 3 курса 321 группы \\
направления 09.03.01 --- Информатика и вычислительная техника \\
Факультета компьютерных наук и информационных технологий \\
Чесакова Максима Евгеньевича
\end{flushleft}

\vspace{313pt}

\begin{center}
    Саратов 2026
\end{center}
%=================================================================================

\textbf{Цель работы:} освоить начальные навыки работы с одноплатными компьютерами: научиться подключаться к плате через последовательный порт и выполнять базовые операции в консольной среде.

\textbf{Ход работы:}
\begin{enumerate}[label=\arabic*.]
    \item Учётная запись в домене лаборатории заведена.
    \item \texttt{sudo} работает. Выполнена команда \texttt{sudo -i}.
    \item Подключён одноплатник к ПК через UART-переходник на рисунке 1.
    \begin{figure}[!ht]
        \centering
        \includegraphics[width=14cm]{../1.jpg}
        \caption{}
    \end{figure}
    \item Выполнено подключение к консоли Linux с помощью утилиты tio:
    
    Для этого было выяснено имя устройства командой \texttt{dmesg | grep -i "tty"} --- USB0.
    
    На рисунке 2 выполнено подключение одноплатника к сети и вставлена microSD-карта с образом операционной системы.
    \begin{figure}[!ht]
        \centering
        \includegraphics[width=14cm]{../2.jpg}
        \caption{}
    \end{figure}
    На рисунке 3 успешное подключение к одноплатнику
    \item Выполнена авторизация в  системе с учетными данными: root/qwe123 (рис. 3).
    \begin{figure}[!ht]
        \centering
        \includegraphics[width=14cm]{../3.jpg}
        \caption{}
    \end{figure}
    \item Выполнение базовых команд:
    \begin{enumerate}[label=6.\arabic*.]
        \item Чтобы узнать текущую нагрузку на ЦП и информацию о нем можно ввести команду \texttt{htop}. Результат приведён на рисунке 4. ЦП загружен на 4\%, используется 60 МБ оперативной памяти из 473 МБ доступной.
        \begin{figure}[!ht]
            \centering
            \includegraphics[width=14cm]{../4.jpg}
            \caption{}
        \end{figure}
        \item Командой \texttt{lsblk} выведена информация о дисках: на одноплатнике один диск объёмом 29,7 ГБ с двумя разделами, один из которых занимает 100 МБ, другой 29,6 ГБ. Подробности на рисунке 5.
        \begin{figure}[!ht]
            \centering
            \includegraphics[width=14cm]{../5.jpg}
            \caption{}
        \end{figure}
        \item Информацию о сетевых интерфейсах можно получить командой \texttt{ifconfig}. В консоли отобразилось два интерфейса: Local Loopback с IP-адресом 127.0.0.1 и wlan0 с IP-адресом 192.168.213.120. Подробности приведены на рисунке 6.
        \begin{figure}[!ht]
            \centering
            \includegraphics[width=14cm]{../6.jpg}
            \caption{}
        \end{figure}
        \item На рисунке 7 изображено создание директории ChesakovME в домашней директории root с помощью команды \texttt{mkdir ChesakovME}. В созданной директории создан файл ChesakovFile.txt с помощью команды \texttt{touch ChesakovFile.txt}. В конец созданного файла была добавлена строка <<Текст Чесакова Максима>> с помощью команды \texttt{echo "Текст Чесакова Максима" >> ChesakovFile.txt}. Далее было выведено содержимое файла в терминал с помощью команды \texttt{cat ChesakovFile.txt}.
        \begin{figure}[!ht]
            \centering
            \includegraphics[width=14cm]{../7.jpg}
            \caption{}
        \end{figure}

        \item Последнее, что было сделано --- это выяснено имя хоста, информация о системе, текущая дата командами \texttt{hostname}, \texttt{hostnamectl}, \texttt{date} соотвественно, что также отображено на рисунке 7.
    \end{enumerate}
\end{enumerate}


\begin{Verbatim}[numbers=left]

\end{Verbatim}









%=================================================================================
\end{document}